\documentclass[a4paper]{article}
\usepackage{polski}
\usepackage[utf8]{inputenc}
\usepackage[left=1cm]{geometry}
\usepackage{float}
\begin{document}
\section{Charakterystyka przejściowa tranzysotra mosfet}
\begin{equation}
I_D = f(U_{GS})
\end{equation}
\begin{equation}
V_{DS} = const.
\end{equation}

\begin{table} [H]
\centering
\begin{tabular} {|c|c|c|c|}
\hline
Dren (D) & SMU2 & V & CONST\\ \hline
Źródło (S) & SMU3 & COM & CONST\\ \hline
Bramka (G) & SMU4 & V & VAR1\\ \hline
\end{tabular}
\caption{Ustawienia kanałów w analizatorze półprzewodników HP4145B.}
\end{table}

\begin{verbatim}
Parametry:
V_DS - stałe
I_DS_MAX
V_GS_START
V_GS_STOP
V_GS_STEP
\end{verbatim}
Komendy GPIB:
\begin{verbatim}
...
\end{verbatim}

%%%%%%%%%%%%%%%%%%%%%%%%%%%%%5		CHARAKTERYSTYKA WYJŚCIOWA	%%%%%%%%%%%%%%%%%%%%%%%%%%%%%
\section{Charakterystyka wyjściowa tranzysotra mosfet}
\begin{equation}
I_D = f(V_{DS})
\end{equation}
\begin{equation}
V_{GS} = const.
\end{equation}

\begin{table} [H]
\centering
\begin{tabular} {|c|c|c|c|}
\hline
Dren (D) & SMU2 & V & VAR1\\ \hline
Źródło (S) & SMU3 & COM & CONST\\ \hline
Bramka (G) & SMU4 & V & CONST\\ \hline
\end{tabular}
\caption{Ustawienia kanałów w analizatorze półprzewodników HP4145B.}
\end{table}

\begin{verbatim}
Parametry:
V_GS - stałe
I_DS_MAX
V_DS_START
V_DS_STOP
V_DS_STEP
\end{verbatim}
Komendy GPIB:
\begin{verbatim}
...
\end{verbatim}

\end{document}
